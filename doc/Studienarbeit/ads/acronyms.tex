%!TEX root = ../dokumentation.tex

\addchap{Abkürzungsverzeichnis}
%nur verwendete Akronyme werden letztlich im Abkürzungsverzeichnis des Dokuments angezeigt
%Verwendung: 
%		\ac{Abk.}   --> fügt die Abkürzung ein, beim ersten Aufruf wird zusätzlich automatisch die ausgeschriebene Version davor eingefügt bzw. in einer Fußnote (hierfür muss in header.tex \usepackage[printonlyused,footnote]{acronym} stehen) dargestellt
%		\acs{Abk.}   -->  fügt die Abkürzung ein
%		\acf{Abk.}   --> fügt die Abkürzung UND die Erklärung ein
%		\acl{Abk.}   --> fügt nur die Erklärung ein
%		\acp{Abk.}  --> gibt Plural aus (angefügtes 's'); das zusätzliche 'p' funktioniert auch bei obigen Befehlen
%	siehe auch: http://golatex.de/wiki/%5Cacronym
%	

%%%%%%%%%%%%%%%% IMPORTIERTER CODE %%%%%%%%%%%%%%%%%%%%%%%%%%%%%%%
% provides \AtBeginEnvironment, \patchcmd and \csdef:

\makeatletter
\newcommand{\acroforeign}[1]{}

% patch the environment to print the foreign definition:
\AtBeginEnvironment{acronym}{%
	\def\acroforeign#1{ (#1)}%
}

% patch the acronym definition to safe the foreign definition:
\expandafter\patchcmd\csname AC@\AC@prefix{}@acro\endcsname
{\begingroup}
{\begingroup\def\acroforeign##1{\csdef{ac@#1@foreign}{##1, }}}
{}
{\fail}

% %   renew the first output to include the foreign definition if given:
\renewcommand*{\@acf}[2][\AC@linebreakpenalty]{%
	\ifAC@footnote
	\acsfont{\csname ac@#2@foreign\endcsname\AC@acs{#2}}%
	\footnote{\AC@placelabel{#2}\AC@acl{#2}{}}%
	\else
	\acffont{%
		\AC@placelabel{#2}\AC@acl{#2}%
		\nolinebreak[#1] %
		\acfsfont{(\acsfont{\csname ac@#2@foreign\endcsname\AC@acs{#2}})}%
	}%
	\fi
	\ifAC@starred\else\AC@logged{#2}\fi
}
\makeatother
%%%%%%%%%%%%%%%%%%%%%%%%%% ENDE IMPORTIERTER CODE %%%%%%%%%%%%%%%%%%%%%%

\begin{acronym}[AUTOSAR]
	%\setlength{\itemsep}{-\parsep}
	\acro{CNN}{Convolutional Neural Network}
	\acro{CycleGAN}{Cycle-Consistent Generative Adversarial Network}
	\acro{GAN}{Generative Adversarial Network}
	\acro{GTSRB}{German Traffic Sign Recognition Benchmark}
	\acro{KNN}{Künstliches Neuronales Netz}
	\acro{LSTM}{Long Short-term Memory}
	\acro{PixelRNN}{Pixel Recurrent Neural Network}
	\acro{SSIM}{Index struktureller Ähnlichkeit (eng.: Structural Similarity Index)}
	\acro{SVM}{Support Vector Machine}
\end{acronym}
