%!TEX root = ../dokumentation.tex

\pagestyle{empty}

\begin{abstract}

A traffic sign detection can nowadays be considered a usual part of automotive vehicles. This is underpinned by the fact that, from the year 2024 on, an EU norm makes such systems mandatory for newly manufactured cars. Traffic sign detection software usually leverages machine learning techniques. Resultingly, images of real traffic signs are necessary for the development of such software. Various datasets exist for this purpose which contain traffic signs from different countries. One property of those datasets is that they can be significantly imbalanced. A reason for this the varying commonness of different traffic sign categories. This can have effects on the quality of detection. One solution is to manually balance those datasets by recording the same number of images for all kinds of traffic signs.

However, for a nearly flawless traffic sign detection, the datasets need to contain different edge cases. Otherwise, future fully autonomous cars can misinterpret traffic signs in specific conditions. One solution is to extend those datasets with real images that are taken in specific weather conditions or where the recognition could be impaired by other factors. This includes vandalism as well as invalid traffic signs. \newline
Another possible solution is to artificially generate such 

A considerable variety of 

An abstract is a brief summary of a research article, thesis, review, conference proceeding or any in-depth analysis of a particular subject or discipline, and is often used to help the reader quickly ascertain the paper's purpose. When used, an abstract always appears at the beginning of a manuscript, acting as the point-of-entry for any given scientific paper or patent application. Abstracting and indexing services for various academic disciplines are aimed at compiling a body of literature for that particular subject.

The terms précis or synopsis are used in some publications to refer to the same thing that other publications might call an ``abstract''. In ``management'' reports, an executive summary usually contains more information (and often more sensitive information) than the abstract does.

Quelle: \url{http://en.wikipedia.org/wiki/Abstract_(summary)}

\end{abstract}