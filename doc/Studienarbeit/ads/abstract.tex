%!TEX root = ../dokumentation.tex

\pagestyle{empty}

\iflang{de}{%
% Dieser deutsche Teil wird nur angezeigt, wenn die Sprache auf Deutsch eingestellt ist.
\renewcommand{\abstractname}{\langabstract} % Text für Überschrift

% \begin{otherlanguage}{english} % auskommentieren, wenn Abstract auf Deutsch sein soll
\begin{abstract}

A traffic sign detection can nowadays be considered a usual part of automotive vehicles. This is underpinned by the fact that, from the year 2024 on, an EU norm makes such systems mandatory for newly manufactured cars. Modern traffic sign detection software usually leverages machine learning techniques. Resultingly, images of real traffic signs are necessary for the development of such software. Various datasets exist for this purpose which contain traffic signs from different countries. One property of those datasets is that they can be significantly imbalanced. A reason for this the varying commonness of different traffic sign categories. Artificially generating such traffic sign images is a possible solution to mitigate this. Some papers already implement this. They base their machine learning models on recent types of generative modelling. \newline This work implements a similar approach. It uses a Cycle-Consistent Generative Adversarial Network to generate images of specific german traffic signs. In contrary to a previous publication, this work uses a Cycle-Consistent Generative Adversarial Network with a U-Net architecture and compares it to a ResNet architecture. The results show that the U-Net based model has a more stable training. Furthermore, a traffic sign classification model shows better performance on the test set when trained with images generated by the U-Net based model than with images generated by the ResNet based model. The question whether this only applies to this specific dataset and model implementation or whether this is a generalizable result is outside the scope of this work. \newline
In addition to this, this work implements functions for artificially augmenting generated traffic sign images. This should simulate external influences such as weather conditions or invalid traffic signs. The idea being that future research can enhance these augmentation techniques. This could be used to simulate edge cases for traffic sign detection software and thus reduce the need for real world data. The implemented augmentations include snow, motion blur and traffic signs that are marked as invalid.
      
\end{abstract}
% \end{otherlanguage} % auskommentieren, wenn Abstract auf Deutsch sein soll
}