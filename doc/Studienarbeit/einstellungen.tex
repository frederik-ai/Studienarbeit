%%%%%%%%%%%%%%%%%%%%%%%%%%%%%%%%%%%%%%%%%%%%%%%%%%%%%%%%%%%%%%%%%%%%%%%%%%%%%%%
%                                   Einstellungen
%
% Hier können alle relevanten Einstellungen für diese Arbeit gesetzt werden.
% Dazu gehören Angaben u.a. über den Autor sowie Formatierungen.
%
%
%%%%%%%%%%%%%%%%%%%%%%%%%%%%%%%%%%%%%%%%%%%%%%%%%%%%%%%%%%%%%%%%%%%%%%%%%%%%%%%


%%%%%%%%%%%%%%%%%%%%%%%%%%%%%%%%%%%% Sprache %%%%%%%%%%%%%%%%%%%%%%%%%%%%%%%%%%%
%% Aktuell sind Deutsch und Englisch unterstützt.
%% Es werden nicht nur alle vom Dokument erzeugten Texte in
%% der entsprechenden Sprache angezeigt, sondern auch weitere
%% Aspekte angepasst, wie z.B. die Anführungszeichen und
%% Datumsformate.
\setzesprache{de} % oder en
%%%%%%%%%%%%%%%%%%%%%%%%%%%%%%%%%%%%%%%%%%%%%%%%%%%%%%%%%%%%%%%%%%%%%%%%%%%%%%%%

%%%%%%%%%%%%%%%%%%%%%%%%%%%%%%%%%%% Angaben  %%%%%%%%%%%%%%%%%%%%%%%%%%%%%%%%%%%
%% Die meisten der folgenden Daten werden auf dem
%% Deckblatt angezeigt, einige auch im weiteren Verlauf
%% des Dokuments.
\setzematrikelnr{6526552}
\setzekurs{TINF20ITA}
\setzetitel{Generierung künstlicher Trainingsdaten für die Straßenschilderkennung in Fahrzeugen mittels Generative Adversarial Networks}
\setzedatumAbgabe{08.06.2023}
\setzefirma{}
\setzefirmenort{}
\setzeabgabeort{Stuttgart}
\setzestudiengang{Informatik}
\setzedhbw{Stuttgart}
\setzebetreuer{Prof. Dr. Monika Kochanowski}
\setzezeitraum{24.10.2022 - 08.06.2023}
\setzearbeit{Studienarbeit}
\setzeautor{Frederik Esau}
%%%%%%%%%%%%%%%%%%%%%%%%%%%%%%%%%%%%%%%%%%%%%%%%%%%%%%%%%%%%%%%%%%%%%%%%%%%%%%%%

%%%%%%%%%%%%%%%%%%%%%%%%%%%% Literaturverzeichnis %%%%%%%%%%%%%%%%%%%%%%%%%%%%%%
%% Bei Fehlern während der Verarbeitung bitte in ads/header.tex bei der
%% Einbindung des Pakets biblatex (ungefähr ab Zeile 110,
%% einmal für jede Sprache), biber in bibtex ändern.
\newcommand{\ladeliteratur}{%
\addbibresource{bibliography.bib}
%\addbibresource{weitereDatei.bib}
}
%% Zitierstil
%% siehe: http://ctan.mirrorcatalogs.com/macros/latex/contrib/biblatex/doc/biblatex.pdf (3.3.1 Citation Styles)
%% mögliche Werte z.B numeric-comp, alphabetic, authoryear
\setzezitierstil{ieee}
%%%%%%%%%%%%%%%%%%%%%%%%%%%%%%%%%%%%%%%%%%%%%%%%%%%%%%%%%%%%%%%%%%%%%%%%%%%%%%%%

%%%%%%%%%%%%%%%%%%%%%%%%%%%%%%%%% Layout %%%%%%%%%%%%%%%%%%%%%%%%%%%%%%%%%%%%%%%
%% Verschiedene Schriftarten
% laut nag Warnung: palatino obsolete, use mathpazo, helvet (option scaled=.95), courier instead
\setzeschriftart{libertine} % palatino oder goudysans, lmodern, libertine

%% Paket um Textteile drehen zu können
%\usepackage{rotating}
%% Paket um Seite im Querformat anzuzeigen
%\usepackage{lscape}

%% Seitenränder
\setzeseitenrand{2.5cm}

%% Abstand vor Kapitelüberschriften zum oberen Seitenrand
\setzekapitelabstand{20pt}

%% Spaltenabstand
\setzespaltenabstand{10pt}
%%Zeilenabstand innerhalb einer Tabelle
\setzezeilenabstand{1.5}
%%%%%%%%%%%%%%%%%%%%%%%%%%%%%%%%%%%%%%%%%%%%%%%%%%%%%%%%%%%%%%%%%%%%%%%%%%%%%%%%

%%%%%%%%%%%%%%%%%%%%%%%%%%%%% Verschiedenes  %%%%%%%%%%%%%%%%%%%%%%%%%%%%%%%%%%%
%% Farben (Angabe in HTML-Notation mit großen Buchstaben)
\newcommand{\ladefarben}{%
	\definecolor{LinkColor}{HTML}{00007A}
	\definecolor{ListingBackground}{HTML}{FCF7DE}
}
%% Mathematikpakete benutzen (Pakete aktivieren)
%\usepackage{amsmath}
%\usepackage{amssymb}

%% Programmiersprachen Highlighting (Listings)
\newcommand{\listingsettings}{%
	\lstset{%
		language=Python,			% Standardsprache des Quellcodes
		numbers=left,			% Zeilennummern links
		stepnumber=1,			% Jede Zeile nummerieren.
		numbersep=5pt,			% 5pt Abstand zum Quellcode
		numberstyle=\tiny,		% Zeichengrösse 'tiny' für die Nummern.
		breaklines=true,		% Zeilen umbrechen wenn notwendig.
		breakautoindent=true,	% Nach dem Zeilenumbruch Zeile einrücken.
		postbreak=\space,		% Bei Leerzeichen umbrechen.
		tabsize=2,				% Tabulatorgrösse 2
		basicstyle=\ttfamily\footnotesize, % Nichtproportionale Schrift, klein für den Quellcode
		showspaces=false,		% Leerzeichen nicht anzeigen.
		showstringspaces=false,	% Leerzeichen auch in Strings ('') nicht anzeigen.
		extendedchars=true,		% Alle Zeichen vom Latin1 Zeichensatz anzeigen.
		captionpos=b,			% sets the caption-position to bottom
		backgroundcolor=\color{ListingBackground}, % Hintergrundfarbe des Quellcodes setzen.
		xleftmargin=0pt,		% Rand links
		xrightmargin=0pt,		% Rand rechts
		frame=single,			% Rahmen an
		frameround=ffff,
		rulecolor=\color{darkgray},	% Rahmenfarbe
		fillcolor=\color{ListingBackground},
		keywordstyle=\color[rgb]{0.133,0.133,0.6},
		commentstyle=\color[rgb]{0.133,0.545,0.133},
		stringstyle=\color[rgb]{0.627,0.126,0.941}
	}
}
%%%%%%%%%%%%%%%%%%%%%%%%%%%%%%%%%%%%%%%%%%%%%%%%%%%%%%%%%%%%%%%%%%%%%%%%%%%%%%%%

%%%%%%%%%%%%%%%%%%%%%%%%%%%%%%%% Eigenes %%%%%%%%%%%%%%%%%%%%%%%%%%%%%%%%%%%%%%%
%% Hier können Ergänzungen zur Präambel vorgenommen werden (eigene Pakete, Einstellungen)
\newcommand*\dif{\mathop{}\!\mathrm{d}}
\usepackage{booktabs}
\usepackage{xfrac}
\usepackage{subcaption}
\usepackage{amsmath}
\usepackage{amssymb}
\usepackage{mathtools}
\usepackage{listings}
\usepackage[newfloat]{minted}
\usepackage{caption}
\newenvironment{code}{\captionsetup{type=listing}}{}
\SetupFloatingEnvironment{listing}{name=Listing}
\captionsetup[listing]{skip=-3pt}
\usepackage{color}
\usepackage{pgfplots}
\usepackage{tablefootnote}
\usepackage{multirow}
\usepackage{booktabs}
\definecolor{LightGray}{gray}{0.95}

% Directory tree
\usepackage[edges]{forest}
\definecolor{folderbg}{RGB}{124,166,198}
\definecolor{folderborder}{RGB}{110,144,169}

\def\Size{4pt}
\tikzset{
      folder/.pic={
        \filldraw[draw=folderborder,top color=folderbg!50,bottom color=folderbg]
          (-1.05*\Size,0.2\Size+5pt) rectangle ++(.75*\Size,-0.2\Size-5pt);  
        \filldraw[draw=folderborder,top color=folderbg!50,bottom color=folderbg]
          (-1.15*\Size,-\Size) rectangle (1.15*\Size,\Size);
      }
    }

%minted
%\inputminted{python}{tflexer.py}
%\newminted[tfcode]{python}{
%	fontsize=\footnotesize, 
%	linenos, 
%	breaklines, 
%	autogobble,
%	frame=lines,
%	autogobble
%	lexer=TFLexer
%}

% chapter settings
\usepackage[T1]{fontenc}
\usepackage{kpfonts}

%\definecolor{chapterlinecolor}{RGB}{136,136,136} % define the color

\addtokomafont{chapter}{\fontfamily{jkpss}\selectfont\normalfont\huge\bfseries\color{black}}
\addtokomafont{chapterprefix}{\fontfamily{jkpss}\selectfont\normalfont\Large\bfseries}
\renewcommand*\chapterformat{%
  {\huge\bfseries\selectfont\thechapter}\enskip%
  %\hspace{0.1em}
  \raisebox{-.3\baselineskip}{\color{black}\rule{1.5pt}{\baselineskip}}\enskip%
  %\hspace{0.1em}
}