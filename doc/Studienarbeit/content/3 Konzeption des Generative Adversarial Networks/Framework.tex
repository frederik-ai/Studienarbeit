\section{Framework}

Seit Version 2.0 unterstützt TensorFlow die \emph{EagerExecution}. Diese wird für die Implementierung der Studienarbeit genutzt. \cite{learn-tensorflow}

\dots
Eine Konvention bei Tensorflow ist, es folgendermaßen zu importieren: \mintinline{python}{import tensorflow as tf}. Rufen Codeblöcke in dieser Studienarbeit Tensorflow-Funktionen auf, werden sie deshalb als \mintinline{python}{tf.some_function()} referenziert.
\dots

\paragraph{Tensorflow Addons}
\paragraph{Tensorflow Graphics}
Erklären: Wieso benutze ich, wo möglich, Tensorflow Graphics statt OpenCV? Wieso gibt es Tensorflow Graphics überhaupt?
\paragraph{OpenCV}
Erwähnen: OpenCV wird nur da benutzt, wo keine Tensorflow funktionen verwendet werden können. Beeinträchtigt die Performance.