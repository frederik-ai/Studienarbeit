\chapter{Augmentation der generierten Bilder}
\label{chap:5}

Wie bereits erwähnt ist das Ziel dieser Studienarbeit nicht alleine die Generierung von Straßenschild-Bildern. Zusätzlich soll die Arbeit einige der in Kapitel \ref{chap:stand-der-technik-strassenschilderkennung} genannten Probleme für die Straßenschilderkennung simulieren. Das Skript \mintinline{python}{generate.py} erlaubt deshalb die Methoden der Augmentierung, die bereits in Tabelle \label{tab:generate-cli} aufgeführt sind. Diese sind eine \textbf{Bewegungsunschärfe} das Hinzufügen von \textbf{Schnee} und das \textbf{markieren von Schildern als ungültig}. Diese Augmentierung von generierten Bildern implementiert ein Modul \mintinline{python}{utils.image_augmentation}. Dieses Kapitel soll genauer auf diese Funktionen eingehen.

\section{Bewegungsunschärfe}
Verwackelte Bilder können insbesondere dann entstehen, wenn sich das Fahrzeug mit einer hohen Geschwindigkeit bewegt. Hierbei entsteht eine Bewegungsunschärfe. Dies tritt bei einer Kamera auf, wenn sich das Bild während der Belichtungszeit deutlich verändert. Wenn also die Geschwindigkeit des Fahrzeugs groß ist im Vergleich zur Belichtungszeit. Da sich das fotografiererte Objekt dazu unterschiedlichen Zeitpunkten an verschiedenen Positionen im Bild befindet, erscheint es als verschwommen. Während bei der Straßenschilderkennung eine Bewegungsunschärfe die Erkennung erschwert, existieren Bereiche, in denen Fotografen absichtlich versuchen sie zu erzeugen. Dazu zählt beispielsweise die Sportfotographie, in der Objekte schneller erscheinen, wenn sie verschwommen zu sehen sind. Die Literatur gibt jedoch Hinweise darauf, dass es nicht trivial sei, die Bewegungsunschärfe mit einer Kamera exakt zu steuern. Beispielsweise wenn die Unschärfe eine bestimmte Intensität haben soll. Aus diesem Grund besteht Interesse daran, Bewegungsunschärfe computergestützt zu erzeugen. \cite{motion-blur}

Ähnlich zu der Funktionsweise von \acp{CNN} basiert die küstliche Erzeugung von Bewegungsunschärfe auf einer Faltung. Die Idee ist, jeden Pixelwert durch einen Durchschnitt der umliegenden Pixelwerte zu ersetzen. Dabei jedoch nicht radial, sondern linear entlang einer bestimmten Richtung. Die Bewegungsunschärfe in dieser Studienarbeit ist entweder horizontal, vertikal oder diagonal. Je mehr Pixel in die Berechnung des Durchschnitts einbezogen werden, desto stärker erscheint die Bewegungsunschärfe. Es ergeben sich dadurch drei Arten von Faltmatrizen: \cite{motion-blur}

\begin{figure}
	\centering
	\begin{tikzpicture}[baseline=(current bounding box.center)]
		\matrix (A) [nodes=draw,column sep=-0.2mm, minimum size=6mm] {
			\node{0}; & \node{0}; & \node{0}; \\
			\node{1}; & \node{1}; & \node{1}; \\
			\node{0}; & \node{0}; & \node{0}; \\
		};
		\matrix (B) [nodes=draw,column sep=-0.2mm, minimum size=6mm, right= of A] {
			\node{0}; & \node{1}; & \node{0}; \\
			\node{0}; & \node{1}; & \node{0}; \\
			\node{0}; & \node{1}; & \node{0}; \\
		};
		\matrix (C) [nodes=draw,column sep=-0.2mm, minimum size=6mm, right = of B] {
			\node{1}; & \node{0}; & \node{0}; \\
			\node{0}; & \node{1}; & \node{0}; \\
			\node{0}; & \node{0}; & \node{1}; \\
		};
	\end{tikzpicture}
	\caption{Horizontale, vertikale und diagonale Faltmatrix}
\end{figure}

Die Größe der Faltmatrix gibt die Stärke der Unschärfe an. Bei einer größeren Faltmatrix fließen nämlich mehr Pixel in die Berechnung des Durchschnitts ein. In dem Modul \mintinline{python}{utils.image_augmentation} existiert die Funktion \mintinline{python}{apply_motion_blur} um auf einen einzelnen Bild-Tensor der Stufe drei eine Bewegungsunschärfe auszuführen. Zusätzlich besitzt die Funktion verschiedene Parameter zur Steuerung der Intensität und der Richtung des Effekts. Die Faltmatrix wird in Numpy erstellt. Die diagonale Faltmatrix fügt 1en auf der Hauptdiagonalen hinzu. Deshalb handelt es sich dabei um eine Einheitsmatrix \emph{(engl.: identity matrix)}. Diese kann mit der folgenden Zeile Code erstellt werden:

\begin{code}
  \begin{minted}{python}
  kernel = np.identity(kernel_size)
  \end{minted}
\end{code}

Wobei \emph{kernel} englisch für Faltmatrix ist und die \emph{kernel\_size} die Größe der Matrix in der Diagonalen angibt. 

Darauf eingehen, dass man das auch mit einer Fourier Transformation umgesetzen kann

Eine Bewegungsunschärfe kann mittels einer Faltung des Bildes mit einer Faltmatrix realisiert werden.

\section{Ungültige Straßenschilder}
In Kapitel \ref{chap:stand-der-technik-strassenschilderkennung} ist bereits beschrieben, dass als ungültig markierte Schilder offenbar eine Herausforderung für heutige Straßenschilderkennungen darstellen können. Das Ziel dieser Studienarbeit ist, solche Fälle simulieren zu können. Aus diesem Grund ist dieser Anwendungsfall implementiert.

Beispiele für durch dieses Projekt erzeugte, ungültige Straßenschilder sind in der nachfolgenden Abbildung dargestellt:

Bei der Umsetzung bieten sich zwei Möglichkeiten. Zum einen kann das \ac{CycleGAN} darauf trainiert werden, solche Bilder eigenständig zu generieren. Dafür würden Trainingsdaten benötigt, die solche Schilder zeigen. Der Datensatz müsste somit um weitere Bilder ergänzt werden. Weitere reale Bilder hinzuzufügen ist nicht ohne weiteres möglich. Es wäre jedoch auch denkbar, bereits vorhandene TRainingsbilder mit einer Bildbearbeitungssoftware so anzupassen, dass die ungültige Schilder zeigen.

Eine weitere Möglichkeit ist, das Kreuz, das die Ungültigkeit eines Schildes markiert, nachträglich in die generierten Bilder einzufügen. Die Schwierigkeit ist hierbei, dass die Straßenschilder zufällig rotiert und skaliert sind. Das Kreuz muss so transformiert werden, dass es sich stest zentral und mit einer angepassten Rotation auf dem Schild befindet.

\section{Schnee}
Die Generierung von Schnee erfolgt in mehreren Schritten:
\begin{enumerate}
  \item Erstelle ein Bild, das aus zufälligen schwarzen und weißen Pixeln besteht. Die Anzahl an weißen Pixel soll kleiner sein als die der schwarzen Pixel.
  \item Führe auf dem Bild ein Gaußsches Weichzeichen aus. Hierdurch verschmieren die einzelnen weißen Pixel zu größeren Punkten.
  \item Führe auf dem Bild eine Bewegungsunschärfe aus. Dadurch wird die Bewegung der Schneeflocken entlang einer Windrichtung simuliert.
  \item Mache den schwarzen Hintergrund transparent und füge das erstellte Bild auf ein generiertes Straßenschild-Bild ein.
\end{enumerate}

\textbf{TODO: Datei einbinden und so linenumbers fixen}
\begin{code}
\label{code:snow-delcaration}
\begin{minted}{python}
def add_snow(img_tensor, snow_intensity, motion_blur_intensity, motion_blur_direction, p_snowflake_min=0.02, p_snowflake_max=0.5):
\end{minted}
\captionof{listing}{Hizufügen von Schnee: Funktionsdeklaration}
\end{code}
