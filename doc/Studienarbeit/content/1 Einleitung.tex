\chapter{Einleitung}
\acused{engl.}
\acused{DHBW}

Während in großen Teilen des letzten Jahrhunderts Innovationen in der Fahrzeugentwicklung vor allem im Bereich der mechanischen Leistung und Effizienz stattgefunden haben, erwartet man zukünftige Verbesserungen im Automobil besonders softwareseitig. Unter anderem im Bereich der Fahrerassistenzsysteme und bezüglich autonomer Fahrzeuge. \cite{Staron2021}

Ein solches Fahrerassistenzsystem, das auch für autonome Fahrzeuge eine Rolle spielt, ist die automatische Erkennung von Straßenschildern. In nicht-autonomen Fahrzeugen unterstützt das System Fahrzeugführende, indem es auf geltende Verkehrsregeln aufmerksam macht. Das ist beispielsweise relevant, wenn der Fahrzeugführende Verkehrsschilder übersieht oder absichtlich missachtet. In autonomen Fahrzeugen ist Software zur Straßenschilderkennung eine der Grundlagen für die Navigation, da die Entscheidungen der Fahrzeug-Software unter anderem auf den durch Straßenschilder kommunizierten Verkehrsvorschriften basieren. \cite{traffic-sign-detection-review-2014}

\section{Problemstellung}

Systeme zur Straßenschilderkennung nutzen mitunter maschinelles Lernen. Entwickelnde führen der Software reale Bilder von Straßenschildern zu. Anhand dessen lernt das System, Bilder von Straßenschildern zu klassifizieren, die es zuvor nicht gesehen hat. Die Qualität der Klassifizierung hängt dabei unter anderem von der Menge und Qualität der Bilddaten ab. \cite{traffic-sign-detection-review-2014}

Aus diesem Grund existieren verschiedene Datensätze, die Bilder von Straßenschildern aus unterschiedlichen Ländern enthalten \cite{GTSRB}. Eine Eigenschaft solcher Datensätze ist, dass sie ungleichmäßig verteilte Daten enthalten können. Das liegt daran, dass bestimmte Arten von Straßenschildern, wie etwa Geschwindigkeitsbegrenzungen, häufiger im Straßenverkehr vorkommen als andere. Ungleichmäßig verteilte Daten können dazu führen, dass das System bestimmte Arten von Schildern fehlerhaft klassifiziert. Weiterhin existieren verschiedene Grenzfälle, die die Qualität der Klassifizierung beeinträchtigen können. Dazu zählen unter anderem bestimmte Wetterbedingungen wie etwa Schnee und Nebel. Idealerweise sollte ein System zur Straßenschilderkennung auch mit selten auftretenden Grenzfällen trainiert werden. Dafür müssen die Datensätze Bilder enthalten, die solche Fälle zeigen. Das ist allerdings, je nach Auftrittswahrscheinlichkeit des Grenzfalls, mit einem erhöhten Aufwand verbunden.

Diese Studienarbeit soll deshalb eine Methode implementieren, die es ermöglicht, Trainingsbilder für die Straßenschilderkennung künstlich zu erzeugen. Solche computergenerierten Bilder sollen die Qualität der Klassifizierung verbessern, indem sie die Datensätze um weitere Bilder ergänzen. Auch kann hierdurch die Verteilung der Trainingsdaten gezielt gesteuert und ausbalanciert werden. Außerdem soll die Implementierung dieser Studienarbeit verschiedene Grenzfälle der Straßenschilderkennung simulieren.

\section{Vorgehensweise}

Zunächst prüft diese Arbeit den aktuellen Stand der Technik von serienmäßig eingesetzter Software zur Straßenschilderkennung. Insbesondere wird darauf Wert gelegt, unter welchen Bedingungen die Straßenschilderkennung fehlerhaft arbeitet. Das sind Fälle, die für diese Studienarbeit eine primäre Rolle spielen.

Anschließend erfolgt ein Überblick, welche Methoden zur computergestützen Generierung von Bildern existieren. Relevant sind hier mitunter bereits existierende Veröffentlichungen, die sich mit der Generierung von Bildern für die Straßenschilderkennung beschäftigen.

Aufbauend darauf wird die Umsetzung der Studienarbeit entworfen und implementiert. Die anschließende Evaluation prüft die Qualität der generieren Bilder. Darauf basiert eine Bewertung, ob sich die Bilder als Trainingsdaten für die Straßenschilderkennung eignen können.

\section{Ziel der Arbeit}
\label{chap:ziel-der-arbeit}

Damit die Bilder einen Mehrwert für die Straßenschilderkennung bieten, müssen sie vor allem fotorealistisch sein. Das ist deshalb eines der konkreten Ziele, die die Evaluation prüft. Dafür erfolgt ein indirekter Vergleich von realen mit künstlich erzeugten Bildern. Außerdem soll die Studienarbeit zeigen, dass Bilder, die Grenzfälle simulieren, Algorithmen zur Straßenschilderkennung zu Fehlinterpretationen verleiten können. Es ist insbesondere eine quantitative Beurteilung relevant, damit die Implementierung dieser Studienarbeit mit anderen Arbeiten vergleichbar ist. 

Die generierten Bilder sollen dabei das Symbol eines einzelnen Straßenschilds zeigen sowie eine geringfügige Menge an Hintergrund. Eine künstliche Generierung von mehreren Schildern pro Bild oder von vollständigen Fahrsituationen ist nicht Ziel dieser Arbeit.