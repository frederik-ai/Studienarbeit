\chapter{Einleitung}
Während in großen Teilen des letzten Jahrhunderts Innovationen in der Fahrzeugentwicklung vor allem im Bereich der mechanischen Leistung und Effizienz stattgefunden haben, erwartet man zukünftige Verbesserungen im Automobil besonders softwareseitig. Unter anderem im Bereich der Fahrerassistenzsysteme und bezüglich autonomer Fahrzeuge. \cite{Staron2021}

Ein solches Fahrerassistenzsystem, das auch für autonome Fahrzeuge eine Rolle spielt, ist die automatische Erkennung von Straßenschildern. In nicht-autonomen Fahrzeugen unterstützt das System Fahrzeugführende, indem es auf geltende Verkehrsregeln aufmerksam macht. Das ist beispielsweise relevant, wenn der Fahrzeugführende Verkehrsschilder übersieht oder absichtlich missachtet. In autonomen Fahrzeugen ist Software zur Straßenschilderkennung eine der Grundlagen für die Navigation, da die Entscheidungen des Fahrzeugs auf den geltenden Verkehrsvorschriften basieren. \cite{traffic-sign-detection-review-2014}

\section{Problemstellung}

Systeme zur Straßenschilderkennung basieren mitunter auf dem maschinellen Lernen. Entwickelnde führen der Software reale Bilder von Straßenschildern zu. Anhand dessen lernt das System, Bilder von Straßenschildern zu klassifizieren, die es zuvor nicht gesehen hat. Die Qualität der Klassifizierung hängt dabei unter anderem von der Menge und Qualität der Bilddaten ab. \cite{traffic-sign-detection-review-2014}

Aus diesem Grund existieren verschiedene Datensätze, die Bilder von Straßenschildern aus unterschiedlichen Ländern enthalten \cite{GTSRB}. Eine Eigenschaft solcher Datensätze ist, dass sie ungleichmäßig verteilte Daten enthalten können. Das liegt daran, dass bestimmte Arten von Straßenschildern, wie etwa Geschwindigkeitsbegrenzungen, häufiger im Straßenverkehr vorkommen als andere. Ungleichmäßig verteilte Daten können dazu führen, dass das System bestimmte Arten von Schildern fehlerhaft klassifiziert. Weiterhin existieren verschiedene Grenzfälle, die die Qualität der Klassifizierung beeinträchtigen können. Dazu zählen unter anderem bestimmte Wetterbedingungen wie etwa Schnee und Nebel. Idealerweise sollte ein System zur Straßenschilderkennung auch mit selten auftretenden Grenzfällen trainiert werden. Dafür müssen die Datensätze Bilder enthalten, die solche Fälle zeigen. Dies ist, je nach Wahrscheinlichkeit des Grenzfalls, mit einem erhöhten Aufwand verbunden.

Diese Studienarbeit soll deshalb eine Methode implementieren, die es ermöglicht, Trainingsbilder für die Straßenschilderkennung künstlich zu erzeugen. Solche computergenerierten Bilder sollen die Qualität der Klassifizierung verbessern, indem sie die Datensätze um weitere Bilder ergänzen. Auch kann hierdurch die Verteilung der Trainingsdaten gezielt gesteuert und ausbalanciert werden. Außerdem soll die Implementierung dieser Studienarbeit Grenzfälle der Straßenschilderkennung simulieren.

\section{Vorgehensweise}

Zunächst prüft diese Arbeit den aktuellen Stand der Technik von serienmäßig eingesetzter Software zur Straßenschilderkennung. Insbesondere wird darauf Wert gelegt, unter welchen Bedingungen die Straßenschilderkennung fehlerhaft arbeitet. Das sind Fälle, die für diese Studienarbeit eine primäre Rolle spielen.

Anschließend wird geprüft, welche Methoden zur computergestützen Generierung von Bildern existieren. Außerdem werden existierende Veröffentlichungen betrachtet, die sich bereits mit der Generierung von Straßenschild-Bildern beschäftigen. 

Aufbauend darauf wird die Generierung entworfen und implementiert. Da es sich um ein Modell handelt, das maschinelles Lernen verwendet, wird das Modell zusätzlich trainiert. Es wird evauliert, wie fotorealistisch die generierten Bilder sind. Anhand dessen kann bewertet werden, ob sich die Bilder als Trainingsdaten für die Straßenschilderkennung eignen.

\section{Ziel der Arbeit}
\label{chap:ziel-der-arbeit}

Damit die Bilder einen Mehrwert für die Straßenschilderkennung bieten können, müssen sie bestimmte Eigenschaften aufweisen. Neben dem Fotorealismus ist es wichtig, dass die Bilder neuartig sind. Ein konkretes Ziel ist deshalb: Die Implementierung dieser Studienarbeit soll dazu in der Lage sein, Bilder von Straßenschildern zu erzeugen, die  fotorealistisch und neuartig sind. Beides kann durch einen Vergleich von realen mit echten Daten erreicht werden. Außerdem soll das Modell eine gewisse Varianz an verschiedenen Bildern erzeugen können. Es ist insbesondere eine quantitative Beurteilung relevant, damit die Implementierung dieser Studienarbeit mit anderen Arbeiten verglichen werden kann.

Die Bilder sollen nicht eine komplette Fahrsituation zeigen, sondern lediglich ein einzelnes Straßenschild auf einem gewissen Hintergrund. Auch verfolgt diese Studienarbeit nicht das Ziel, ein Modell zu entwickeln, das auf diese Weise bereits umfassend in der Industrie eingesetzt werden kann. Stattdessen soll die Implementierung zukünftige Arbeiten auf diesem Gebiet anstoßen, die beispielsweise noch mehr Grenzfälle simulieren können.