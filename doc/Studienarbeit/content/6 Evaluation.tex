\chapter{Evaluation}
Die Aufgabe der Evaluation ist prinzipiell folgende: Es soll gemessen werden, wie ähnlich die Wahrscheinlichkeitsverteilung der generierten Bilder zu der Verteilung der Trainingsbilder ist. Außerdem soll bestimmt werden, wie realistisch die generierten Bilder sind.

\label{chap:Evaluation}
\section{Evaluation der Generierung}

%\begin{table}[H]
%	\centering
%	\begin{tabular}{|l|c|c|c|}
%	\hline
%	Trainingsdatensatz & Trainingsbilder (\#) & Trainingsg.\tablefootnote{Genauigkeit auf den Trainingsdaten} (\%) & Epochen (\#) \\ \hline \hline
%	Präparierter GTSRB & 4.554 & 100 & 20 \\ \hline
%   Gesamte Trainingsdaten & 5.685 & 100 & 10 \\ \hline \hline
%   U-Net & 4.300 & 99,60 & 16 \\ \hline
%	ResNet (6 residual blocks) & 4.300  & 100 & 20 \\ \hline
%   ResNet (9 residual blocks) & 4.300 & 99,33 & 20 \\ \hline \hline
%   Piktogramme & 43 & 100 & 20 \\ \hline
%   Augmentierte Piktogramme & 4.300 & 99,65 & 20
%    \\ \hline \hline
%	Gemischt & & & \\ \hline
%	\end{tabular}
%	\caption{Training eines VGG16 Klassifikators mit unterschiedlichen Trainingsdaten}
%\end{table}

\begin{table}[H]
	\centering
	\begin{tabular}{|l|c|c|}
	\hline
	Trainingsdatensatz & Trainingsbilder (\#) & Testgenauigkeit (\%) \\ \hline \hline
	Präparierter GTSRB & 4.554 & 82 \\ \hline
   Gesamte Trainingsdaten & 5.685 & 83 \\ \hline \hline
   U-Net & 4.300 & 62 \\ \hline
	ResNet (6 residual blocks) & 4.300 & 46 \\ \hline
   ResNet (9 residual blocks) & 4.300 & 53 \\ \hline \hline
	Augmentierte Piktogramme & 4.300 & 19 \\ \hline
   Piktogramme & 43 & 12 \\ \hline \hline
	Gemischt & \\ \hline
	\end{tabular}
	\caption{Ergebnisse des Trainings eines VGG16 Klassifikators}
\end{table}

\section{Evaluation der Augmentierung}

\section{Verbesserungsmöglichkeiten}
\begin{itemize}
   \item Bestimmte Funktionen in TensorFlow Graphen umwandeln; Dadurch Performance der Implementierung verbessern
\end{itemize}