\chapter{Implementierung und Training}

Die Implementierung des \ac{CycleGAN} basiert auf den bisher beschriebenen Entscheidungen. In einem Ordner \mintinline{python}{src} befinden sich alle Dateien der Implementierung. Hier existieren neben Dateien für das Training und die Anwendung des Modells auch eine Konfigurationsdatei und verschiedene Hilfsfunktionen. Die Ordnerstruktur der Implementierung ist im Anhang abgebildet. Im wesentlichen geht dieses Kapitel auf folgende Bestandteile ein:
\begin{itemize}
   \item Die Konfigurationsdatei für das Projekt befindet sich im Pfad \mintinline{python}{config/config.toml}
   \item Selbst-definierte Hilfsfunktionen befinden sich im Python-Modul \mintinline{python}{utils}
   \item Die Datei \mintinline{python}{model.py} implementiert das \ac{CycleGAN} Modell
   \item Das Modell wird mittels des Python-Skripts \mintinline{python}{train.py} trainiert
   \item Das Modell kann mittels des Python-Skripts \mintinline{python}{generate.py} zum Generieren von Bildern genutzt werden
\end{itemize}

Das \ac{CycleGAN} ist als Klasse implementiert und lässt sich somit innerhalb der Skripte instanziieren. Die Konfigurationsdatei ist eine \acs{TOML}-Datei. \ac{TOML} ist eine Konfigurationssprache, die Daten als Schlüssel-Werte-Paare speichert. Die Konfigurationsdatei ist in die Kategorien \mintinline{python}{paths}, \mintinline{python}{model} und \mintinline{python}{training} unterteilt. Anwendende können hier die Pfade zu den Trainingsdaten oder aber Parameter wie \mintinline{python}{batch_size} oder \mintinline{python}{number_of_epochs} angeben.

\acused{TOML}

\section{Modell}
Das Modell ist in der Datei \mintinline{python}{model.py} implementiert. Innerhalb dieser Datei existiert eine Klasse \ac{CycleGAN}, die im wesentlichen folgende Methoden besitzt:

%\begin{description}
%   \item[\lstinline[python]{__init__(self, config)}] This method does something.
%   \item[method2] This method does something else.
%   \item[method3] This method does yet another thing.
% \end{description}

\begin{table}[H]
   \centering
   \begin{tabular}{|l|l|}
   \hline
   \textbf{Methode}                    & \textbf{Aufgabe} \\ \hline
   \_\_init\_\_ & Initialisieren der Attribute mittels der Konfigurationsdatei \\ \hline
   compile  & Kompilieren der Generator- und Diskriminatormodelle      \\ \hline
   generate & Generieren eines einzelnen Batches von Bildern  \\ \hline
   fit & Trainieren des \acp{CycleGAN} über mehrere Epochen \\ \hline
   train\_step & Durchführen eines einzelnen Trainingsschritts \\ \hline
   restore\_...\_checkpoint\_...\tablefootnote{Vollständiger Methodenname: restore\_latest\_checkpoint\_if\_exists} & Laden der aktuell gespeicherten Parameter des Modells \\
   \hline
   \end{tabular}
   \caption{Auswahl an Methoden aus der CycleGAN Klasse}
\end{table}



Das Modell objektorientiert zu implementieren scheint entgegen üblicher Konventionen zu stehen. 

Um schnellere Ergebnisse zu erzielen, damit sich vor allem auf die Optimierung der Implementierung und der Augmentierung der generierten Bilder konzentriert werden kann, basiert der Inhalt dieser Klasse zum Teil auf einem Beispiel von TensorFlow \cite{cyclegan-tutorial}. Dieses Beispiel beinhaltet eine Implementierung eines \acp{CycleGAN}. Der Quelltext dieses Beispiels wird für die Umsetzung des \acp{CycleGAN} dieser Studienarbeit als Basis genommen und verändert, beziehungsweise erweitert. Es sei hierbei betont, dass jede Codezeile, die identisch mit dem Code aus dem Beispiel ist, vorher hinterfragt und bezüglich ihrer Sinnhaftigkeit für diesen Anwendungsfall getestet worden ist. Darauf soll explizit in diesem Kapitel eingegangen werden.

\section{Generator- und Diskriminatorarchitekturen}
Wie bereits erwähnt, besteht ein \ac{CycleGAN} aus vier \acp{KNN}: Generator X, Generator Y, Diskriminator X, Diskriminator Y. Initiale Versuche, diese \acp{KNN} eigenständig zu implementieren, haben zu keinen Ergebnissen geführt. In dem Sinne, dass die Modelle nicht gelernt haben. Auch unter der Verwendung von Veröffentlichungen, die Richtlinien über die Architektur solcher Netze geben. In der Konfigurationsdatei kann deshalb zwischen zwei vorgefertigten Architekturen gewählt werden: \emph{U-Net} und \emph{ResNet}.

\subsection{U-Net}
Es existiert eine Veröffentlichung, die \emph{Conditional \acp{GAN}} für die Bild-zu-Bild Generierung verwendet. Das Modell der Veröffentlichung ist mitunter dazu in der Lage, schwarz-weiß-Bilder einzufärben oder aus Zeichnungen ein möglichst fotorealistisches Bild zu erzeugen. Der Generator verwendet eine abgewandelte Form eines \emph{U-Net}. Dies ist eine Architektur, die das Eingangsbild zunächst auf eine geringere Anzahl an Neuronen Kodiert und anschließend zu einem neuen Bild dekodiert. Die Architektur ähnelt somit konzeptionell der eines Autoencoders. Der Diskriminator verwendet eine Architektur, die sich \emph{PatchGAN} nennt. Die Besonderheit ist hierbei, dass der Diskriminator weniger einzelne Pixel betrachtet, sondern Teile eines Bilds als echt oder unecht klassifizieren kann.

Der Quellcode der Veröffentlichung ist unter dem Namen \emph{Pix2Pix} bekannt und wird auch so importiert. 

\section{Datenaugmentierung}
In Kapitel \ref{chap:3-datenaugmentation} ist das Vorgehen für die Datenaugmentierung beschrieben. Wie bereits gezeigt, nutzt beispielsweise die \mintinline{python}{fit}-Methode die Datenaugmentierung, um dem Generator $G$ die Größe und Perspektive des Straßenschilds vorzugeben. Hier soll auf die konkrete Implementierung dessen eingegangen werden. Zur Augmentierung müssen die Piktogramme der Straßenschilder zufällig rotiert und skaliert werden. Hierzu dient die Bibliothek Tensorflow Graphics.

Die Augmentierung ist in der Datei \mintinline{python}{utils/preprocess_image.py} implementiert. Die Funktion, die hierbei aus der CycleGAN Klasse aufgerufen wird, ist \mintinline{python}{randomly_transform_image_batch}. Das Listing \ref{lst:pictogram-augmentation} im Anhang zeigt die vollständige Implementierung. Die Funktion erhält einen vierdimensionalen Tensor \mintinline{python}{img_tensor_batch} als Eingang. Dieser Tensor beinhaltet den Batch an Bildern, der transformiert werden soll. Zunächst skaliert die Funktion zufällig den Inhalt dieser Bilder. Anschließend führt sie darauf eine zufällige dreidimensionale Rotation aus und gibt die transformierten Bilder zurück. Was sie ebenfalls zurückgibt, sind die Listen \mintinline{python}{content_sizes} und \mintinline{python}{rotation_matrices}. Die Anzahl an Elementen der Listen entspricht der Größe des übergebenen Batches, also der Anzahl an transformierten Bildern. Hierdurch kann die aufrufende Funktion für jedes Bild identifizieren, welche Zufallswerte für die Transformation generiert wurden. Dies kann genutzt werden, um die Transformation zu replizieren. Genutzt wird das in Kapitel \ref{chap:5}, um Bilder von als ungültig markierten Schildern zu erzeugen.

\subsection{Skalierung}
Die Skalierung des Bildinhalts besteht aus mehreren Schritten. Die Funktion \mintinline{python}{randomly_transform_image_batch} generiert zunächst mittels Numpy eine Liste an zufälligen \mintinline{python}{content_sizes}. Danach skaliert die Funktion die Piktogramme der Straßenschilder auf die in \mintinline{python}{content_sizes} gespeicherten Pixelgrößen mittels der Hilfsfunktion \mintinline{python}{resize_content_of_img}. Das ist in Listing \ref{lst:content-scaling} gezeigt.

\begin{code}
   \begin{minted}{python}
content_sizes_tmp = content_sizes[:]
transformed_imgs = tf.map_fn(
   lambda img: resize_content_of_img(
      img, target_size, content_sizes_tmp.pop(0)), 
   img_tensor_batch)
\end{minted}
   \captionof{listing}{Skalieren der Bild-Tensoren}
   \label{lst:content-scaling}
\end{code}

Als ersten Schritt dupliziert die Funktion die Liste \mintinline{python}{content_sizes} in der Variable \mintinline{python}{content_sizes_tmp}. Anschließend skaliert die Funktion die Bilder. Dazu nutzt sie die von TensorFlow bereitgestellte Funktion \mintinline{python}{tf.map_fn}. Diese Funktion erhält als Parameter einen Tensor der Stufe $n$ und eine Funktion. Sie führt die Funktion auf jedem Element der Stufe $n-1$ des Tensors aus. Beispielsweise auf jedem Bild eines Batches von Bildern. In diesem Fall übergeben wir einen Tensor der Stufe vier und der Form \emph{(Batch Größe, Breite, Höhe, Anzahl Farbkanäle)} an die Funktion \mintinline{python}{tf.map_fn}. Die Funktion erstellt daraus eine Menge an Tensoren der Stufe drei, wobei jeder dieser Tensoren ein Bild mit der Form \emph{(Breite, Höhe, Anzahl Farbkanäle)} ist.
Auf jedem Bild führt die Funktion \mintinline{python}{tf.map_fn} anschließend die Funktion \mintinline{python}{resize_content_of_img} aus. Letztere Funktion erhält ein Bild und die Zielgröße, auf die der Inhalt des Bilds skaliert werden soll. Anschließend fügt die Funktion \mintinline{python}{tf.map_fn} die Bilder wieder zu einem einzelnen Tensor der stufe vier zusammen und gibt ihn zurück. \cite{tf-map-fn}

Es wäre ebenso möglich, über die einzelnen Bild-Tensoren des Batches mittels einer \mintinline{python}|for|-Schleife zu iterieren. Die Dokumentation von \mintinline{python}{tf.map_fn} gibt jedoch explizit an, dass sie eine parallele Ausführung ermöglicht. Das ist hier der ausschlaggebende Vorteil gegenüber einer \mintinline{python}|for|-Schleife. Es ist jedoch dennoch weniger performant als eine Funktion zu verwenden, die eine einzelne Operation vektorisiert auf dem gesamten Tensor ausführt. Warum die Funtionen dennoch mit \mintinline{python}{tf.map_fn} arbeitet, statt alle Bilder des Batchs gleichzeitig zu transformieren, soll der folgende Abschnitt klären. \cite{tf-map-fn}

Die Funktion \mintinline{python}{resize_content_of_img} verwendet zwei Funktionen aus dem TensorFlow Framework: Die Funktion \mintinline{python}{tf.image.resize} um das Bild zu skalieren und die Funktion \mintinline{python}{tf.image.resize_with_crop_or_pad} um das Bild zurück auf die ursprüngliche Größe zu bringen. Wird das Piktogramm verkleinert, dann fügt letztere Fuktion einen weißen Rand um das Piktogramm hinzu. Wird es hingegen vergrößert, schneidet die Funktion die Pixel ab, die über die Zielgröße hinausgehen. Obwohl es möglich wäre, den vierdimensionalen Tensor an beide TensorFlow Funktionen zu übergeben, um alle Bilder gleichzeitig zu skalieren, erhält \mintinline{python}{resize_content_of_img} lediglich dreidimensionale Tensoren, sprich einzelne Bilder, als Parameter. Das hängt damit zusammen, dass die Piktogramme in einem Batch unterschiedliche Skalierungen besitzen sollen. Die TensorFlow Funktionen sind jedoch nur dazu in der Lage, eine bestimmte Skalierung auf allen Bilder des Batches auszuführen.

Die Rolle der Liste \mintinline{python}{content_sizes_tmp} ist, dass während jedem Durchlauf der Funktion \mintinline{python}{tf.map_fn}

\todo[inline]{Rolle der Liste content\_sizes\_tmp erklären}

\subsection{Rotation}
Der zweite Schritt der Augmentation ist, dass das Modell die Piktogramme rotiert. Es existieren Rotationsmatrizen, die Rotationen mittels eulerscher Winkel beschreiben. Die Drehungen um die x-, y- und z-Achse besitzen jeweils eigene Rotationsmatrizen $R_x$, $R_y$ und $R_z$. Der Aufbau dieser Matrizen ist der Literatur entnommen \cite{math-primer}. Um daraus eine einzelne Rotationsmatrix $R$ zu erhalten, werden die Rotationsmatrizen miteinander multipliziert. Dazu dient die Funktion \mintinline{python}{create_rotation_matrix}. Sie erhält die drei Winkel $\alpha_x$, $\alpha_y$ und $\alpha_z$ als Parameter und gibt eine einzelne Rotationsmatrix $R$ zurück.

Bevor die Funktion \mintinline{python}{randomly_transform_image_batch} die Funktion \mintinline{python}{create_rotation_matrix} aufruft, muss sie ausgehend davon zunächst zufällige Winkel $(\alpha_x, \alpha_y, \alpha_z)$ erzeugen. An dieser Stelle entstammen die zufälligen Winkel jedoch nicht einer Gleichverteilung, sondern einer gaußschen Normalverteilung. Das hängt damit zusammen, dass die Piktogramme in den meisten fällen nur leicht rotiert sein sollen. Nur ein vergleichsweise geringer Prozentsatz der Piktgoramme soll stark Augmentiert sein. Dies soll in etwa nachbilden, aus welchen Perspektiven die Straßenschilder im Trainingsdatensatz aufgenommen sind. Listing \ref{lst:alpha-z-value} zeigt die zufällige Generierung der Rotationswinkel $\alpha_z$:

\begin{code}
   \begin{minted}{python}
alpha_z_values = np.random.normal(loc=0.0, scale=3.5, 
                                 size=batch_size)
\end{minted}
   \captionof{listing}{Zufällige Generierung der Rotationswinkel $\alpha_z$}
   \label{lst:alpha-z-value}
   \end{code}

Der Parameter \mintinline{python}{loc} gibt den Erwartungswert der Winkel an, während \mintinline{python}{scale} die Standardabweichung setzt. Durch das Angeben der \mintinline{python}{batch_size} wird deutlich, dass die Numpy Funktin hier nicht nur einen Winkel erzeugt wird, sondern ein \emph{Numpy Array} das für jedes Piktogramm einen zufälligen Winkel enthält. Die Implementierung ist somit hier vektorisiert. Im Mittel liegen die Winkel der Rotation bei 0,0\textdegree. Der Wert für die Standardabweichung ist empirisch bestimmt, da die Werte für die Winkel nicht den tatsächlichen Winkeln in Grad entsprechen. Die Funktion \mintinline{python}{randomly_transform_image_batch} erzeugt die Winkel $\alpha_y$ und $\alpha_x$ analog hierzu, mit dem Unterschied, dass sie dort andere Standardabweichungen (\mintinline{python}{scale}) als Parameter setzt. Dass eine gaussche Normalverteilung verwendet wird, bedeutet, dass die meisten Piktogramme zu einer Aufnahme aus der Frontalperspektive führen. Der Großteil der Rotationswinkel ist demnach vergleichsweise gering. Einige wenige Schilder hingegen sind stärker rotiert. Würde die Funktion eine Gleichverteilung zur Erzeugung der Winkel verwenden, wäre der Anteil an starken Rotationen in etwa gleich zu dem Anteil an geringen Rotationen.

Die eigentliche Rotation setzt die TensorFlow Graphics Funktion \mintinline{python}{perspective_transform} um. Sie erhält einen Tensor der Stufe vier an Bildern und einen Tensor der Stufe vier an Rotationsmatrizen. Das bedeutet, dass der gesamte Batch an Bildern samt seiner Rotationsmatrizen übergeben wird. Somit erfolgt die Augmentation der Bilder hier vektorisiert und damit parallel. Die Codezeile \mintinline{python}{transformed_imgs = 1 - transformed_imgs} kehrt vor der Rotation zunächst die Farbwerte des Bilds um. Das ist nötig, da der Hintergrund, den die genannte TensorFlow Funktion erzeugt, schwarz ist. Kehrt man zunächst die Farbwerte um, führt die Rotation aus und setzt die Farbwerte auf ihren ursprünglichen Wert zurück, so wird der schwarze Hintergrund durch einen weißen ersetzt.

\section{Training}
Anwendende können das \ac{CycleGAN} mit dem Skript \mintinline{python}{train.py} trainieren. Prinzipiell besitzt dieses Skript zwei Aufgaben: Es lädt sowohl den Trainingsdatensatz als auch die Piktogramme und ruft die Trainingsfunktion des Modells \ac{CycleGAN} auf.

\subsection{Laden der Datensätze}
Die Konfigurationsdatei enthält in der Kategorie \mintinline{python}{paths} den Eintrag \mintinline{python}{train_data}. Das ist der relative oder absolute Pfad zu dem Trainingsdatensatz. Das Skript \mintinline{python}{train.py} lädt den Datensatz in ein \mintinline{python}{tf.data.Dataset} Objekt. Dafür stellt TensorFlow eine Funktion bereit, mit der ein Datensatz an Bildern aus einem Dateipfad geladen werden kann. Anhand der Ordnerstruktur sortiert die Funktion die Bilder automatisch in ihre Klassen ein. Die Funktion nennt sich \mintinline{python}{load_image_dataset_from_directory} \cite{tf-keras-utils}. In folgendem Listing ist der Teil des Skripts dargestellt, der mittels dieser Funktion den Datensatz lädt.

\begin{code}
   \label{code:train-set-laden}
   \begin{minted}{python}
training_path = config['paths']['train_data']

train_set = tf.keras.utils.image_dataset_from_directory(training_path, batch_size=BATCH_SIZE, image_size=(IMAGE_SIZE, IMAGE_SIZE), labels=None, shuffle=True, crop_to_aspect_ratio=True)

train_set_processed = utils.load_data.normalize_dataset(train_set)
   \end{minted}
   \captionof{listing}{\lstinline{train.py} - Laden des Trainingsdatensatzes}
\end{code}

An die genannte TensorFlow Funktion übergibt das Skript mitunter den Pfad zu den Trainingsdaten, die Batch Größe und die Bildauflösung. Die Auflösung muss deshalb übergeben werden, da die Funktion \mintinline{python}{load_image_dataset_from_directory} alle Bilder auf diese Größe skaliert. Hierzu nutzt die Funktion standardmäßig \emph{bilineare Interpolation}. Dadurch erscheint das Bild nicht als \emph{verpixelt}, sondern fehlende Pixel, die bei der Vergrößerung unweigerlich auftreten, werden durch eine Kombination der benachbarten Pixel aufgefüllt. Dadurch wirkt das Bild statt \emph{verpixelt} eher \emph{verwaschen}. Das \ac{CycleGAN} benötigt die Daten nicht nach ihren Klassen sortiert, da es mit unüberwachtem Lernen arbeitet. Deshalb wird zusätzlich der Parameter \mintinline{python}{labels} auf \mintinline{python}{None} gesetzt.

Durch den nächsten Parameter \mintinline{python}{shuffle} erfolgt die Einstellung, dass der Datensatz zufällig durchmischt werden soll. Abschließend folgt ein entscheidender Parameter, der einer näherer Erläuterung bedarf. Wie bereits in Kapitel \ref{chap:3-Datensatz} beschrieben, besitzen die Trainingsbilder des Datensatzes verschiedene Auflösungen. Das bedeutet, dass Bilder die nicht quadratisch sind, durch die Funktion \mintinline{python}{load_image_dataset_from_directory} verzerrt würden, damit sie in ein quadratisches Seitenverhältnis von 256x256 Pixel passen. Tests haben ergeben, dass die meisten Bilder des Datensatzes nur gerinfügig verzerrt werden. Einige Bilder besitzen jedoch signifikant mehr Pixel in der Höhe als in der Breite oder umgekehrt. Um dafür zu sorgen, dass alle Bilder ohne Verzerrung in das Modell gespeist werden, existiert der Parameter \mintinline{python}{crop_to_aspect_ratio}. Dieser Parameter sorgt dafür, dass das Bild derart zugeschnitten wird, dass es in das angegebene Bildformat passt. Hierbei wird das Bild so zugeschnitten, dass es gerade in das Seitenverhältnis passt. Was stets erhalten bleibt, ist der zentrale Teil des Bilds. Da sich die Straßenschilder in den meisten Bildern mittig befinden, ist dies genau das gewünschte Verhalten.

Was die Funktion \mintinline{python}{load_image_dataset_from_directory} zurückgibt, ist ein Objekt vom Typ \mintinline{python}{tf.data.Dataset}. Es kann somit direkt für die \mintinline{python}{CycleGan.fit}-Methode verwendet werden. Ein letzter Schritt ist, die Bilder zu normalisieren. Dies geschieht mittels der Funktion \mintinline{python}{normalize_dataset} aus dem eigens definierten Modul \mintinline{python}{utils.load_data}. Diese Funktion normalisiert die Pixelwerte der Bilder auf den Bereich von -1 bis 1. Dies ist notwendig, um die Bilder in die \acp{KNN} einzuspeisen.

Damit ist das Laden der Trainingsdaten abgeschlossen. Die Piktogramme werden unter der Verwendung des gleichen Schemas geladen. Es ergibt sich ein \mintinline{python}{tf.data.Dataset} Objekt mit dem Namen \mintinline{python}{pictograms_processed}.

\subsection{Ausführen des Trainings}
Das Training wird vollständig durch die Klasse \mintinline{python}{CycleGan} durchgeführt. Sind sowohl die Piktogramme als auch die Trainingsbilder geladen, sind folgende Codezeilen notwendig, um das Training zu starten:
\begin{code}
    \begin{minted}{python}
cycle_gan = model.CycleGan(config)
cycle_gan.restore_latest_checkpoint_if_exists()
cycle_gan.fit(pictograms_processed, train_set_processed, epochs=config['training']['number_of_epochs'])
    \end{minted}
    \captionof{listing}{\lstinline{train.py} - Laden des Trainingsdatensatzes}
 \end{code}

Das Skript \mintinline{python}{train.py} instanziiert zunächst ein \mintinline{python}{CycleGan} Objekt. Falls vorherige Parameter in einem Checkpoint gespeichert sind, werden diese anschließend geladen. Ansonsten werden die Parameter durch TensorFlow zufällig initialisiert. Abschließend erfolgt der Aufruf der \mintinline{python}{fit}-Methode mit der in der Konfigurationsdatei angegebenen Anzahl an Epochen.

\subsection{Logging}
Ebenfalls bietet die Implementierung die Möglichkeit, den Verlauf der Verlustfunktionen über das Training zu betrachten. Dazu ist in der Implementierung zusätzlicher Code, der dieses sogenannte \emph{Logging} ermöglicht. Das Logging erfolgt mittels der Bibliothek \mintinline{python}{TensorBoard}. Diese Bibliothek ist Teil des TensorFlow Frameworks. Mit TensorBoard ist es ebenso möglich, diesen Verlauf zu visualisieren. Die hierzu notwendigen Konsolenbefehle für jeweils das U-Net- und ResNet-basierte \ac{CycleGAN} zeigt folgendes Listing:
\begin{code}
   \begin{minted}{bash}
$ tensorboard --logdir ./logs/unet
$ tensorboard --logdir ./logs/resnet
   \end{minted}
\end{code}

\section{Trainingsergebnisse}
\label{chap:trainingsergebnisse}
Für das Training der U-Net- und ResNet-basierten \acp{CycleGAN} dienen zwei verschiedene Systeme. Dabei zum einen Google Colab. Hier bietet Google eine kostenfreie Version sowie ein Premium-Abonnement an. In der kostenfreien Version ist jedoch kein Training über Nacht möglich, da das System nach einer etwa zwanzig-minütigen Inaktivität die Verbindung zum Rechner trennt. Unter anderem aus diesem Grund wird das Training zusätzlich auf einem Server der \ac{DHBW} durchgeführt. Dieser besitzt eine Grafikkarte mit 25 Gigabyte Speicher und ist damit leistungsstärker als die bei Google Colab verwendeten Systeme. Dort standen maximal 20 GB zur Verfügung. Die Trainingsdauer pro Epoche ist für die traininerten Modelle in der nachfolgenden Tabelle aufgeführt. Es zeigt sich hier außerdem, dass das U-Net signifikant schneller trainiert als das ResNet, jedoch Checkpoints besitzt, die mehr Speicherplatz verbrauchen. Das bedeutet, dass das Training des U-Net-basierten \ac{CycleGAN} weniger Rechenaufwand benötigt, obwohl es mehr Parameter besitzt.

\begin{table}[H]
   \centering
   \begin{tabular}{lllll}
   \toprule
   & \multicolumn{2}{c}{Trainingsdauer pro Epoche} & \\
   
   \cmidrule(r){2-3}
   
   Modell & Google Colab & \ac{DHBW} Server & Parameter & Checkpoint Größe \\
   \midrule
   U-Net & 30 min.\tablefootnote{Minuten} & 5 min. & 114 Millionen & 1.340.240 Byte \\
   ResNet & 90 min. & 30 min. & 28 Millionen & 331.709 Byte \\
   \bottomrule
    \end{tabular}
    \caption{Vergleich von U-Net und ResNet}
\end{table}

\subsection{U-Net}
Das U-Net-basierte \ac{CycleGAN} verbessert sich während des Trainings nahezu kontinuierlich. Das Modell ist mit einer Anzahl von 200 Epochen trainiert. Der Verlauf der Verlustfunktionen über die Anzahl der Trainingsschritte ist im Anhang in Abbildung \ref{fig:unet-tensorboard} gezeigt. Ein Trainingsschritt entspricht einem Durchlauf der \mintinline{python}{train_step} Funktion des \ac{CycleGAN} über einem Batch.  Es lassen sich verschiedene Dinge aus den Graphen ablesen: Die Verluste der Generatoren scheinen gegen einen Wert zu konvergieren. Die Verluste der Diskriminatoren zeigen hingegen deutliche Schwankungen ohne ein erkennbares Muster. Die Verluste der Generatoren sind beinahe um einen Faktor 10 größer als die der Diskriminatoren. Aus diesem Grund konvergiert der Gesamtverlust des \ac{CycleGAN} gegen einen Wert. Dieser liegt bei 6.

Abbildung \ref{fig:unet-generated-imgs} im Anhang zeigt zudem für die Epochen 1 bis 100, wie sich die Qualität der Generierung mit den Trainingsepochen verbessert. Hier lässt sich außerdem erkennen, dass das Modell durch die Wahl einer Bild-zu-Bild-Übersetzung die Schilder nicht selber erzeugen muss. Verschiedene Trainingsdurchläufe haben ergeben, dass der Identity Loss für das U-Net keinen signifikanten Einfluss hat. Damit kann argumentiert werden, dass er für dieses \ac{CycleGAN} nicht nötig sei.

Die generierten Bilder zeigen verschiedene Hintergründe. Diese Varianz an Hintergründen ist allgemein pro Kategorie von Straßenschild gleich. Somit besitzen beispielsweise alle Schilder der Kategorie Geschwindigkeitsbegrenzung die gleichen Arten von Hintergründen. Die Schilder der Kategorie Aufhebung können allgemein als wenig fotorealistisch bewertet werden. Das ist vermutlich auf die geringere Anzahl an Trainingsdaten für diese Klassen zurückzuführen.

Abbildung \ref{fig:unet-imgs} im Anhang zeigt Bilder, die das U-Net-basierte Modell nach 200 Epochen generiert. Ausgehend von den Erkenntnissen der in Kapitel \ref{chap:vorherige-arbeiten-taiwan} vorgestellte Veröffentlichung könnte sich die Qualität der Generierung über noch mehr Trainingsepochen weiter verbessern. Aus folgenden Gründen ist das U-Net-basierte Modell genau 200 Epochen trainiert:
\begin{itemize}
    \item Die Verlustfunktionen scheinen zu konvergieren.
    \item Für ein weiteres Training muss vermutlich die Lernrate verringert werden
    \item Das Training von 200 Epochen dauert bereits 16 Stunden auf dem System der \ac{DHBW}
    \item U-Net und ResNet sollten für die Evaluation in Kapitel \ref{chap:Evaluation} in etwa die gleiche Anzahl an Epochen trainiert sein
\end{itemize}

\subsection{ResNet}

Für das ResNet-basierte \ac{CycleGAN} zeigen die Verlustfunktionen einen ähnlichen Verlauf wie bei dem U-Net-basierten \ac{CycleGAN}. Aus diesem Grund sind hierfür die Graphen nicht im Anhang abgebildet. Der Unterschied ist, dass der Verlauf der Verlustfunktionen hier eine geringere Aussagekraft für die Qualität der generierten Bilder zu haben scheint. Das Training des ResNet-basierten Modells ist oszillierend, da das \ac{CycleGAN} einige Klassen in nachfolgenden Epochen besser erzeugt, während andere Klassen eine schlechtere Generierungsqualität als davor aufweisen. Um diesem Verhalten entgegenzuwirken, verwendet dieses \ac{CycleGAN}-Modell eine $\mathcal{L}_2$ Verlustfunktion für den Adversarial Loss, während das U-Net-basierte Modell weiterhin eine logarithmische Verlustfunktion verwendet. Auch diese Veränderung der Verlustfunktion, die laut der Literatur das Training stabilisieren kann, behebt das oszillierende Verhalten nicht.

Eine weitere Eigenschaft dieses Modells ist, dass bei 200 Epochen ein Modal Collaps auftritt. Das ist in Abbildung \ref{fig:modal-collaps} dargestellt. Hier und in den folgenden Abbildungen steht die Abkürzung \emph{Ep.} für das Wort \emph{Epochen} und beschreibt damit die Anzahl an Trainingsepochen, aus der das Bild stammt. 
% MODAL COLLAPS
\begin{figure}[H]
    \centering
    \begin{subfigure}[b]{0.12\textwidth}
        \centering
        \includegraphics[height=\textwidth]{../images/4 Implementierung und Training/Generierte Bilder je Epoche/ResNet 9 res blocks/Modal collaps/200 epochs.png}
        \caption{200 Ep.}
    \end{subfigure}
    \hspace{3em}%
    \begin{subfigure}[b]{0.12\textwidth}
        \centering
        \includegraphics[height=\textwidth]{../images/4 Implementierung und Training/Generierte Bilder je Epoche/ResNet 9 res blocks/Modal collaps/200 epochs 2.png}
        \caption{200 Ep.}
    \end{subfigure}
    \hspace{3em}%
    \begin{subfigure}[b]{0.12\textwidth}
        \centering
        \includegraphics[height=\textwidth]{../images/4 Implementierung und Training/Generierte Bilder je Epoche/ResNet 9 res blocks/Modal collaps/200 epochs 3.png}
        \caption{200 Ep.}
    \end{subfigure}
    \hspace{3em}%
    \begin{subfigure}[b]{0.12\textwidth}
        \centering
        \includegraphics[height=\textwidth]{../images/4 Implementierung und Training/Generierte Bilder je Epoche/ResNet 9 res blocks/Modal collaps/200 epochs 4.png}
        \caption{200 Ep.}
    \end{subfigure}
    \hspace{3em}%
    \begin{subfigure}[b]{0.12\textwidth}
    \centering
    \includegraphics[height=\textwidth]{../images/4 Implementierung und Training/Generierte Bilder je Epoche/ResNet 9 res blocks/Modal collaps/200 epochs 6.png}
    \caption{200 Ep.}
\end{subfigure}
        \caption{Modal Collaps des ResNet nach 200 Trainingsepochen}
        \label{fig:modal-collaps}
\end{figure}
Für jedes Piktogramm und jede Perspektive erzeugt das Modell durch den Modal Collaps einen beinahe identischen Hintergrund. Das gibt Hinweise darauf, dass die Generatoren die Diskriminatoren derart überlisten, dass letztere nicht mehr lernen.

Eine Lösung ist, das Training vorzeitig abzubrechen \emph{(engl.: early stopping)}. Diese Lösung wird gewählt. Dabei muss für jede infrage kommende Epoche manuell geprüft werden, welche davon das zufriedenstellendste Ergebnis zeigt. Auch da die Werte der Verlustfunktionen hierauf, wie bereits erwähnt, nur eine begrenzte Aussagekraft haben. Jede der Epochen 120 bis 190 erzeugt eine Bandbreite an Generierungsqualitäten. Abbildung \ref{fig:resnet-gute-bilder} zeigt positiv herausstechende Bilder für verschiedene Epochen, während \ref{fig:resnet-schlechte-bilder} negativ herausstechende Bilder zeigt. Die finale Entscheidung ist, das auf 180 Epochen trainierte Modell zu verwenden.

% RESNET 9 BLOCK GUTE BILDER
\begin{figure}[h]
    \centering
    \begin{subfigure}[b]{0.12\textwidth}
        \centering
        \includegraphics[height=\textwidth]{../images/4 Implementierung und Training/Generierte Bilder je Epoche/ResNet 9 res blocks/Gute Bilder/120 epochs.png}
        \caption{120 Ep.}
    \end{subfigure}
    \hspace{3em}%
    \begin{subfigure}[b]{0.12\textwidth}
        \centering
        \includegraphics[height=\textwidth]{../images/4 Implementierung und Training/Generierte Bilder je Epoche/ResNet 9 res blocks/Gute Bilder/160 epochs.png}
        \caption{160 Ep.}
    \end{subfigure}
    \hspace{3em}%
    \begin{subfigure}[b]{0.12\textwidth}
        \centering
        \includegraphics[height=\textwidth]{../images/4 Implementierung und Training/Generierte Bilder je Epoche/ResNet 9 res blocks/Gute Bilder/170 epochs.png}
        \caption{170 Ep.}
    \end{subfigure}
    \hspace{3em}%
    \begin{subfigure}[b]{0.12\textwidth}
        \centering
        \includegraphics[height=\textwidth]{../images/4 Implementierung und Training/Generierte Bilder je Epoche/ResNet 9 res blocks/Gute Bilder/170 epochs 2.png}
        \caption{170 Ep.}
    \end{subfigure}
    \hspace{3em}%
    \begin{subfigure}[b]{0.12\textwidth}
    \centering
    \includegraphics[height=\textwidth]{../images/4 Implementierung und Training/Generierte Bilder je Epoche/ResNet 9 res blocks/Gute Bilder/180 epochs.png}
    \caption{180 Ep.}
\end{subfigure}
        \caption{Positiv herausstechende Bilder des ResNets verschiedener Epochen}
        \label{fig:resnet-gute-bilder}
\end{figure}

% RESNET 9 BLOCKS SCHLECHTE BILDER
\begin{figure}[h]
    \centering
    \begin{subfigure}[b]{0.12\textwidth}
        \centering
        \includegraphics[height=\textwidth]{../images/4 Implementierung und Training/Generierte Bilder je Epoche/ResNet 9 res blocks/Schlechte Bilder/120 epochs.png}
        \caption{120 Ep.}
    \end{subfigure}
    \hspace{3em}%
    \begin{subfigure}[b]{0.12\textwidth}
        \centering
        \includegraphics[height=\textwidth]{../images/4 Implementierung und Training/Generierte Bilder je Epoche/ResNet 9 res blocks/Schlechte Bilder/140 epochs 2.png}
        \caption{140 Ep.}
    \end{subfigure}
    \hspace{3em}%
    \begin{subfigure}[b]{0.12\textwidth}
        \centering
        \includegraphics[height=\textwidth]{../images/4 Implementierung und Training/Generierte Bilder je Epoche/ResNet 9 res blocks/Schlechte Bilder/140 epochs.png}
        \caption{140 Ep.}
    \end{subfigure}
    \hspace{3em}%
    \begin{subfigure}[b]{0.12\textwidth}
        \centering
        \includegraphics[height=\textwidth]{../images/4 Implementierung und Training/Generierte Bilder je Epoche/ResNet 9 res blocks/Schlechte Bilder/180 epochs 2.png}
        \caption{180 Ep.}
    \end{subfigure}
    \hspace{3em}%
    \begin{subfigure}[b]{0.12\textwidth}
    \centering
    \includegraphics[height=\textwidth]{../images/4 Implementierung und Training/Generierte Bilder je Epoche/ResNet 9 res blocks/Schlechte Bilder/180 epochs.png}
    \caption{180 Ep.}
\end{subfigure}
        \caption{Negativ herausstechende Bilder des ResNets verschiedener Epochen}
        \label{fig:resnet-schlechte-bilder}
\end{figure}

Abbilung \ref{fig:resnet-generated-imgs} im Anhang zeigt Bilder, die das ResNet-basierte Modell nach 180 Epochen generiert.

Eine zweite Lösung im Gegensatz zum vorzeitigen Abbruch des Trainings ist, dass die Generatoren ein anderes ResNet-Modell verwenden. Ein Modell, das verhindern kann, dass die Generatoren die Diskriminatoren vollständig überlisten können. Aus diesem Grund besitzen die Generatoren in einem nächsten Trainingsdurchlauf 6 Residual Blocks, statt der durch die \ac{CycleGAN} Veröffentlichung vorgeschlagene Anzahl an 9 für Bilder der Auflösung 256x256. Dadurch haben die Generatoren eine weniger komplexe Architektur mit einer geringeren Anzahl an lernbaren Parametern. Während hier kein Modal Collaps auftritt, ist das Training weiterhin oszillierend. Auch hier zeigt die Epoche 200 nicht die höchste Generierungsqualität. Bei diesem ResNet-basierten Modell wird deshalb Epoche 150 für die Evaluation in Kapitel \ref{chap:Evaluation} gewählt. Damit soll geprüft werden, welche Anzahl an Residual Blocks sich für das Modell eignet. Die Abbildung \ref{fig:resnet-6-blocks} zeigt Beispielbilder des ResNet-basierten Modells mit 6 Residual Blocks für verschiedene Epochen.

% RESNET 6 BLOCKS
\begin{figure}[H]
    \centering
    \begin{subfigure}[b]{0.12\textwidth}
        \centering
        \includegraphics[height=\textwidth]{../images/4 Implementierung und Training/Generierte Bilder je Epoche/ResNet 6 res blocks/Schlechte Bilder/epoche 140.png}
        \caption{140 Ep.}
    \end{subfigure}
    \hspace{1em}%
    \begin{subfigure}[b]{0.12\textwidth}
        \centering
        \includegraphics[height=\textwidth]{../images/4 Implementierung und Training/Generierte Bilder je Epoche/ResNet 6 res blocks/Gute Bilder/epoch 170.png}
        \caption{170 Ep.}
    \end{subfigure}
    \hspace{1em}%
    \begin{subfigure}[b]{0.12\textwidth}
        \centering
        \includegraphics[height=\textwidth]{../images/4 Implementierung und Training/Generierte Bilder je Epoche/ResNet 6 res blocks/Schlechte Bilder/epoche 180.png}
        \caption{180 Ep.}
    \end{subfigure}
    \hspace{1em}%
    \begin{subfigure}[b]{0.12\textwidth}
    \centering
    \includegraphics[height=\textwidth]{../images/4 Implementierung und Training/Generierte Bilder je Epoche/ResNet 6 res blocks/Gute Bilder/epoche 180 2.png}
    \caption{180 Ep.}
\end{subfigure}
\hspace{1em}%
    \begin{subfigure}[b]{0.12\textwidth}
    \centering
    \includegraphics[height=\textwidth]{../images/4 Implementierung und Training/Generierte Bilder je Epoche/ResNet 6 res blocks/Gute Bilder/epoche 180.png}
    \caption{180 Ep.}
\end{subfigure}
\hspace{1em}%
    \begin{subfigure}[b]{0.12\textwidth}
    \centering
    \includegraphics[height=\textwidth]{../images/4 Implementierung und Training/Generierte Bilder je Epoche/ResNet 6 res blocks/Schlechte Bilder/epoche 190.png}
    \caption{190 Ep.}
\end{subfigure}
        \caption{Beispielbilder des ResNets mit 6 Residual Blocks}
        \label{fig:resnet-6-blocks}
\end{figure}

Generierte Bilder des ResNet-basierten Modells mit 6 Residual Blocks befinden sich unter \href{https://drive.google.com/drive/folders/11gaUErheUYb0WlBPtWhxgCK7mE0URHYI?usp=sharing}{\textbf{dem Link des Datensatzes}} in dem Ordner \mintinline{python}{'Generated Images'}. Hier sind zudem weitere generierte Bilder des ResNet-basierten Modells mit 9 Residual Blocks sowie des U-Net-basierten Modells.

\section{Generierung}
Das Skript \mintinline{python}{generate.py} dient dazu, Bilder von Straßenschildern mittels eines trainierten \ac{CycleGAN} Modells zu generieren. Eine Design-Entscheidung ist hierbei, dass das Skript vollständig mittels der Kommandozeile konfiguriert werden kann. Dies folgt dem Leitprinzip dieser Studienarbeit, dass Anwendende keinen Python-Code anpassen müssen. Das Skript besitzt die in Tabelle \ref{tab:generate-cli} gezeigten Kommandozeilenargumente.
\begin{table}[H]
   \centering
   \begin{tabular}{|l|c|l|}
   \hline
   \textbf{Argument} & \textbf{Parameter} & \textbf{Aufgabe} \\ \hline \hline
   -{}-num-imgs & Ganzzahl & Anzahl der zu generierenden Bilder \\ \hline
   -{}-model & \mintinline{python}{'unet'} oder \mintinline{python}{'resnet'} & Art des Modells \\ \hline
   -{}-motion-blur & - & Fügt Bewegungsunschärfe hinzu \\ \hline
   -{}-make-invalid & - & Markiert Schilder als ungültig \\ \hline
   -{}-snow & - & Fügt Schnee hinzu \\
   \hline
   \end{tabular}
   \caption{Kommandozeilenargumente des Skripts \lstinline[language=python]{generate.py}}
   \label{tab:generate-cli}
\end{table}

Standardmäßig nutzt das Skript \mintinline{python}|generate.py| das Modell aus der Konfigurationsdatei, mit dem Kommandozeilenargument \mintinline{python}|--model| können Anwendende diesen Wert jedoch überschreiben. Die Argumente \mintinline{bash}{--motion-blur}, \mintinline{bash}{--make-invalid} und \mintinline{bash}{--snow} können außerdem miteinander kombiniert werden. Dann erzeugt das Skript mehrere Augmentationen gleichzeitig. Auf die Implementierung dieser Augmentationen geht Kapitel \ref{chap:5} ein.

Beispielhafte Aufrufe des Skripts \mintinline{python}|generate.py| mit verschiedenen Kommandozeilenargumenten zeigt Listing \ref{lst:generate-cli-examples}.
\begin{samepage}
\begin{code}
   \begin{minted}{bash}
      $ python generate.py --num-imgs 10 --motion-blur
      $ python generate.py --num-imgs 10 --model 'resnet' --make-invalid
      $ python generate.py --num-imgs 10 --snow --motion-blur
      $ python generate.py --num-imgs 50 --model 'unet'
   \end{minted}
   \caption{Beispielaufrufe des Skripts \lstinline[language=python]{generate.py}}
   \label{lst:generate-cli-examples}
\end{code}
\end{samepage}

Ein spezieller Anwendungsfall dieses Modells könnte der folgende sein: Anwendende möchten nur für bestimmte Arten von Straßenschildern Bilder generieren, statt für alle 43. Beispielsweise nur für Stopp-Schilder. Das ist insbesondere dann relevant, wenn das Modell dazu genutzt werden soll, einen bestehenden Datensatz auszugleichen. Wenn etwa bestimmte Klassen unterrepräsentiert sind.
Das Skript \mintinline{python}{generate.py} erlaubt explizit das folgende Vorgehen: Anwendende können aus dem Ordner, in dem sich die Bilder der Piktogramme befinden, alle Arten von Straßenschildern löschen, die nicht generiert werden sollen. Befindet sich in dem Ordner beispielsweise nur ein Piktogramm für Stopp-Schilder, dann werden auch nur Bilder von Stopp-Schildern generiert. Dafür kann zum Beispiel ein zweiter Ordner für die Piktogramme angelegt werden, der dann in der \ac{TOML}-Konfigurationsdatei unter dem Wert \mintinline{python}{'pictograms'} innerhalb der Kategorie \mintinline{python}{'paths'} angegeben wird.

In dem Pfad \mintinline{python}{experimental/generate_single_classes.py} existiert ein Skript, dass dieses Vorgehen für alle Klassen automatisiert. Es ruft das Skript \mintinline{python}|generate.py| nacheinander für jede der 43 Arten von Straßenschildern auf und sortiert die generierten Bilder in separate Ordner ein.